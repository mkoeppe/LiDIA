%%%%%%%%%%%%%%%%%%%%%%%%%%%%%%%%%%%%%%%%%%%%%%%%%%%%%%%%%%%%%%%%%
%%
%%  overview.tex       LiDIA documentation
%%
%%  This file contains comments on the design of LiDIA
%%
%%  Copyright   (c)   1996   by LiDIA-Group
%%
%%  Authors: Johannes Buchmann, Ingrid Biehl, Thomas Papanikolaou
%%


\newcommand{\Pari}{\textbf{Pari}\xspace}
\newcommand{\Kant}{\textbf{Kant}\xspace}
\newcommand{\Simath}{\textbf{Simath}\xspace}
\newcommand{\Maple}{\textbf{Maple}\xspace}
\newcommand{\Mathematica}{\textbf{Mathematica}\xspace}


\chapter{Overview}

This chapter describes the overall organization of the \LiDIA system.


\section{Introduction}\label{sect:Pref}

In early 1994 our research group in computational number theory at the Department of Computer
Science at the Universit\"at des Saarlandes, Germany, was working on implementations of
algorithms for factoring integers, determining discrete logarithms in finite fields, counting
points on elliptic curves over finite fields, etc.  In those implementations three different
multiprecision integer packages were used.  The code written for those implementations was hard
to read and hardly documented as well.  Therefore, many basic routines were written over and
over again, often not very efficiently.

Because of this we decided to organize all the software in a library which we called
\Index{\LiDIA}.  The organization of this task, the development of the general concept of the
library and the implementation of the basic components like the \emph{kernel}, the
\emph{interface} and the majority of the components for doing arithmetic were designed and
implemented by Thomas Papanikolaou.

Since we find that \emph{object oriented programming} is appropriate for implementing
mathematical algorithms and since C++ belongs to the most accepted programming languages in
scientific computing, we decided to use C++ as the \emph{implementation language} for \LiDIA.
To guarantee easy \emph{portability} of \LiDIA we decided to have a very small machine dependent
kernel in \LiDIA.  That kernel currently contains the multiprecision integer arithmetic and a
memory manager.  All \LiDIA objects are implemented in C++ and compiled with many different
compilers on various architectures.

Although not in the public domain, \LiDIA can be used freely for non commercial purposes (see
copyright notice, page \pageref{chap:copyright}).


\section{Design notes}

It is a serious problem to decide which multiprecision integer package and which memory manager
should be used in \LiDIA.  There are competing multiprecision integer packages, for example
\begin{itemize}
\item the Berkeley MP package;
\item the GNU MP package by Torbj\"orn Granlund (\url{http://www.swox.com/gmp/});
\item the cln package by Bruno Haible (\url{http://clisp.cons.org/~haible/packages-cln.html});
\item the piologie package from HiPiLib (\url{http://www.hipilib.de/piologie.htm});
\item the libI package by Ralf Dentzer (available from
  \url{http://www.informatik.tu-darmstadt.de/TI/LiDIA/});
\item the freeLIP package, formerly known as LIP, by Arjen Lenstra, now maintained by Paul
  Leyland (\url{ftp://ftp.ox.ac.uk/pub/math/freelip/});
\item the BIGNUM routines of the openssl library (\url{http://www.openssl.org/}).
\end{itemize}
Some of those packages are more efficient on one architecture and some on others\footnote{As of
  March 2001, GNU MP seems to be superior, while libI and freelip seem to be deprecated.}.
Also, new architectures lead very fast to new multiprecision packages.  We decided to make
\LiDIA independent of a particular kernel and to make it easy to replace the \LiDIA kernel.  To
achieve this, we separate the kernel from the application programs by an \emph{interface} in
which the declarations, operators, functions, and procedures dealing with multiprecision
integers and the memory management are standardized.  All \LiDIA classes use the kernel through
that interface.  They never use the kernel functions directly.  Whenever new kernel packages are
used only the interface has to be changed appropriately but none of the \LiDIA classes has to be
altered.  In the \LiDIAVersion package we use the libI package and its memory manager as the
default kernel.

We try to develop extremely efficient code for the algorithms we offer in the \LiDIA system.
Therefore, \LiDIA is designed in such a way, that a user can decide in which level of
abstraction one implements the programs.  Moreover we try to keep the dependency of parts of the
library on each other to a minimum and to define interfaces precisely in order to be able to
replace modules by more efficient modules in an efficient way.

\section{Structure of \LiDIA}

The \LiDIA library has 5 levels.

The lowest level contains the \Index{kernel}, which contains a \Index{multi-precision} integer
arithmetic and a \Index{memory manager}.  Both components are currently written in C to attain
maximum speed\footnote{However, it is possible to use any other implementation language in the
  kernel, as long as this language can be called from C++.}.

\attentionIII The multi-precision integer arithmetic has been an integral part of \LiDIA, i.e.
the sources have been part of \LiDIA's source tree and the multi-precision arithmetic library
has been rolled into the \LiDIA library.  \textbf{Beginning with release version 2.1 the
  multi-precision arithmetic library is no longer a part of \LiDIA.  In order to use \LiDIA, the
  multi-precision arithmetic library has to be installed, and the user must explicitely link
  that library to his applications!}

In \LiDIA's kernel it is possible to use various multiprecision integer packages since the
functions of the kernel are never called directly by any application program of \LiDIA but only
through the interface on the second level.  Therefore, any package supporting the whole
functionality of the interface can be used in the kernel.  Currently, we support the
multiprecision integer packages libI, freelip, GNU MP, cln and piologie.

The second \LiDIA level is the \emph{interface} through which C++ applications on higher
levels have access to the kernel.  By that interface declarations, operators, functions and
procedures dealing with multiprecision integers and memory management are standardized.  \LiDIA
applications always use those standards.  They never call the underlying kernel directly.  The
interface for the multiprecision integer arithmetic is realized by the class \code{bigint} and
the interface for the general memory manager is implemented in the class \emph{gmm}.

The third level of \LiDIA contains all non parameterized \LiDIA applications, the so-called
\Index{simple classes}.  For example, on the third level there is a class \code{bigrational}
which implements the arithmetic of rational numbers.

On the fourth \LiDIA level there are the \Index{parameterized classes} (also called
\Index{container classes}).  These classes are designed to solve computational problems which are
very similar in many concrete applications.  In the current version this level includes generic
implementations of hash tables, vectors, matrices, power series, and univariate polynomials.

The highest level of \LiDIA includes the user interface which consists of an online
documentation tool and the \emph{interpreter} lc \index{\lc} which makes the \LiDIA
functions available for interactive use.

\label{LidFig}
\begin{center}
  \setlength\unitlength{1mm}
\begin{picture}(142,96)
  \sffamily\small
% Background coloring commands.  These use graphics \specials, but do not
% print anything else, thus plain DVI previewers can simply ignore them.
  \setlength\fboxsep{0pt}
  \put(0,0){\fcolorbox{black}{lightgray}{\phantom{\rule{142mm}{96mm}}}}
  \multiput(9,3)(0,18){5}{\colorbox{darkgray}{\phantom{\rule{130mm}{12mm}}}}
  \multiput(6,6)(0,18){5}{\colorbox{white}{\phantom{\rule{130mm}{12mm}}}}
% Textboxes
  \put(6,6){\framebox(130,12){\parbox{120mm}{{kernel}\hfill\texttt{(libI, gmp, cln, piologie, mm)}}}}
  \put(6,24){\framebox(130,12){\parbox{120mm}{{interfaces}\hfill\texttt{(bigint, gmm)}}}}
  \put(6,42){\framebox(130,12){\parbox{120mm}{{simple classes}\hfill\texttt{(bigrational, bigfloat, \dots)}}}}
  \put(6,60){\framebox(130,12){\parbox{120mm}{{parameterized classes}\hfill\texttt{(vector, matrix, power series, \dots)}}}}
  \put(6,78){\framebox(130,12){\parbox{120mm}{{user interfaces}\hfill\texttt{(lchelp, lc)}}}}
% Double-headed arrows
  \multiput(71,21)(0,18){4}{\vector(0,1){3}}
  \multiput(71,21)(0,18){4}{\vector(0,-1){3}}
% Dashed lines (even without epic)
  \multiput(53,6)(0,3){4}{\line(0,1){.75}}
  \multiput(53,18)(0,-3){4}{\line(0,-1){.75}}
  \multiput(53,24)(0,3){4}{\line(0,1){.75}}
  \multiput(53,36)(0,-3){4}{\line(0,-1){.75}}
  \multiput(53,42)(0,3){4}{\line(0,1){.75}}
  \multiput(53,54)(0,-3){4}{\line(0,-1){.75}}
  \multiput(53,60)(0,3){4}{\line(0,1){.75}}
  \multiput(53,72)(0,-3){4}{\line(0,-1){.75}}
  \multiput(53,78)(0,3){4}{\line(0,1){.75}}
  \multiput(53,90)(0,-3){4}{\line(0,-1){.75}}
\end{picture}


  Fig.~1: The levels of \LiDIA
\end{center}


\section{Packages}

Since 1994, \LiDIA has become quite a big collection of routines and classes, and many people
have contributed to its functionality.  As the development of LiDIA has shifted towards more and
more specialized areas, we have decided to slightly reorganise its design in order to
\begin{itemize}
\item remain maintainable and to
\item allow an easy integration of new routines.
\end{itemize}
Therefore, we have divided \LiDIA into several packages; the standard packages are the
following:
\begin{itemize}
\item The \LiDIA base package includes all the basic arithmetic and most of the template
  classes.  This package is always required.
\item The \LiDIA FF package contains the classes which deal with finite fields.
\item The \LiDIA LA package contains the classes which deal with linear algebra over the ring of
  rational integers and over $\ZmZ$.  It requires the \LiDIA FF package.
\item The \LiDIA LT package contains the classes which deal with lattices over the integers and
  over the reals.  It requires the \LiDIA FF package, and the \LiDIA LA package.
\item The \LiDIA NF package contains the classes which deal with quadratic number fields, and
  number fields of higher degree.  It requires the \LiDIA FF package, the \LiDIA LA package, and
  the \LiDIA LT package.
\item The \LiDIA EC package contains the classes which deal with elliptic curves over the
  integers and over finite fields.  It requires the \LiDIA FF package, and the \LiDIA LA
  package.
\item The \LiDIA ECO package contains classes and code to count the rational points of elliptic
  curves over $\GF(p)$ and $\GF(2^n)$.  It requires the \LiDIA EC package, and
  accordingly the \LiDIA FF package and the \LiDIA LA package.
\item The \LiDIA GEC package contains classes and code to generate
  cryptographically strong elliptic curves and a prime proofer class.
  It requires the \LiDIA ECO package
  and the prerequisites thereof, \LiDIA FF, \LiDIA LA, and \LiDIA EC, as well
  as \LiDIA NF.
\end{itemize}


\section{Future developments}

In the future we will expand the set of classes and algorithms by offering additional packages.
Methods which are available in \LiDIAVersion will also be available in future releases, so as to
ensure compatibility of user implemented algorithms with new versions of \LiDIA.  If you develop
algorithms of general interest using \LiDIA, we would appreciate it if you let us include them
in the \LiDIA distribution (possibly as a new package) or send information about your
implementation to other \LiDIA users.

The list of currently available packages can be found at the \LiDIA WWW Home Page of \LiDIA:
\begin{quote}
  \url{http://www.informatik.tu-darmstadt.de/TI/LiDIA}
\end{quote}


\section{Addresses of the \LiDIA-group}

After having picked up the source of \LiDIAVersion, for example by anonymous ftp via
\begin{quote}
  \url{ftp://ftp.informatik.tu-darmstadt.de/pub/TI/systems/LiDIA},
\end{quote}
you should subscribe to the \LiDIA mailing lists \url{LiDIA@cdc.informatik.tu-darmstadt.de} and
\url{LiDIA-announce@cdc.informatik.tu-darmstadt.de}.  You can do so by sending email to
\url{LiDIA-request@cdc.informatik.tu-darmstadt.de} and
\url{LiDIA-announce-request@cdc.informatik.tu-darmstadt.de}, respectivly.  Put the word
``subscribe'' in the body.

Alternatively, you can use the Web-interface at
\url{http://www.cdc.informatik.tu-darmstadt.de/mailman/listinfo/} to subscribe to the \LiDIA
mailing lists.

\attentionI Former \LiDIA manuals requested \LiDIA users to send an email to
\url{lidia@cdc.informatik.tu-darmstadt.de}.  This is no longer desired.  Instead, subscribe to
\url{LiDIA} and \url{LiDIA-announce} to get the latest news on \LiDIA.

If you think you have found a bug in \LiDIA (or incomplete/incorrect documentation), please
report to
\begin{quote}
  \url{LiDIA-bugs@cdc.informatik.tu-darmstadt.de}
\end{quote}
Note that we will probably not be able to reconstruct a bug without a detailed
description.  We need to know at least the \LiDIA version, the operating system, and the
compiler that you were using.


\section{Acknowledgements}

The following people have contributed to \LiDIA (in alphabetical order):

Detlef Anton, Werner Backes, Harald Baier, Jutta Bartholomes, Franz-Dieter
Berger, Ingrid Biehl, Emre Binisik, Oliver Braun, Christian Cornelssen, John
Cremona, Sascha Demetrio, Thomas F.~Denny, Roland Dreier, Friedrich
Eisenbrand, Dan Everett, Tobias Hahn, Safuat Hamdy, Jochen Hechler, Birgit
Henhapl, Michael J.~Jacobson, Jr., Patrick Kirsch, Thorsten Lauer, Frank
J.~Lehmann, Christoph Ludwig, Markus Maurer, Andreas M\"uller, Volker
M\"uller, Stefan Neis, Thomas Papanikolaou, Thomas Pfahler, Thorsten
Rottsch\"afer, Dirk Schramm, Victor Shoup, Nigel Smart, Thomas Sosnowski, Anja
Steih, Patrick Theobald, Oliver van Sprang, Damian Weber, and
Susanne Wetzel. 

Special thanks to Thomas Papanikolaou for originating \LiDIA and Prof.~Johannes Buchmann for
supporting the project.
